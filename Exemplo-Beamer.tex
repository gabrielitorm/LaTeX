
% Exemplo apresentação beamer
% By: Gabrielito Menezes
% Informações sobre o documento:

\documentclass{beamer}        % Definir a classe do documento

% Pacotes que serão utilizados:
\usepackage[utf8]{inputenc}   % Codificação dos caracteres de entrada
\usepackage[brazilian]{babel} % Idioma em pt-br

% Definir estilo:
\usetheme{Madrid}			 	% Estilo da apresentação (tema)

% Informações:  
\title{Título da Apresentação}  % Título da apresentação
\subtitle{Subtítulo} 			% Subtítulo
\author{Autor}				 	% Nome do autor
\institute{Nome da instituição} % Inserir a isntistuição 
\date{\today} % Inserir data de hoje! Para remover a data basta "\date{}"

\begin{document}
	
	% A primeiro slide (Apresentação)
	\begin{frame}
		\titlepage 
	\end{frame}
	
		
	% A estrutura do frame
	\begin{frame}{Estrutura da apresentação}
		\tableofcontents
	\end{frame}


% Listas de frames (slides)
\section{Lista não ordenada no Beamer}
\begin{frame}{Listas no Beamer}
	
	Esta é uma lista não ordenada:
	\begin{itemize}
		\item Item 1
		\item Item 2
		\item Item 3
	\end{itemize}
	
\end{frame}

% Slide lista não ordenada 2
\begin{frame}{Listas no Beamer}
	
	\begin{itemize}
		\item Item 1
		\item Item 2
		\item Item 3
	\end{itemize}
	
\end{frame}

% Slide Lista ordenada 1
\section{Lista ordenada Beamer}
\begin{frame}{Lista ordenada Beamer}
	
		Esta é uma lista ordenada:
	\begin{enumerate}
		\item Item 1
		\item Item 2
		\item Item 3
	\end{enumerate}

\end{frame} 

% Slide Lista ordenada 2
\begin{frame}{Lista ordenada Beamer}
	
	\begin{enumerate}
		\item Item 1
		\item Item 2
		\item Item 3
	\end{enumerate}
	
\end{frame} 

\end{document}